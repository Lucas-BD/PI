%% USPSC-Introducao.tex

% ----------------------------------------------------------
% Introdução (exemplo de capítulo sem numeração, mas presente no Sumário)
% ----------------------------------------------------------
\chapter[Introdução]{Introdução}
\label{Introdução}


O monitoramento das causas comuns de morte é uma preocupação antiga, \citeonline{malta2007lista}. destaca que a construção de uma lista de mortes evitáveis é um primeiro passo para o monitoramento que, então, pode orientar as políticas públicas para a prevenção, como um exemplo pratico \cite{marson2010mortes} realizaram um estudo no Hospital Universitário de Londrina e achou uma possível relação entre mortes por trauma e a não aplicação de diretrizes de atendimento, dado este fato se conclui que uma parcela de mortes poderia ser evitada caso o problema fosse identificado e as diretrizes não fossem ignoradas.

Outro desafio é o registro dos dados nos sistemas, tanto pela não inclusão como pela qualidade dos dados, \citeonline{muzy2021analise} destaca que os dados disponibilizados no SIM muitas vezes são incompletos ou defasados e o preenchimento dos dados ée feito de forma ambígua.

O objetivo deste relatório é propor uma solução para o problema do monitoramento dado os dados disponíveis pelo sistema SIM, o processo é feito de forma simples ao articular soluções de fácil acesso. 
