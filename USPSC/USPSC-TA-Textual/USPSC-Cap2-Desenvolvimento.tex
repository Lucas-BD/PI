%% USPSC-Cap2-Desenvolvimento.tex 

% ---
% Este capítulo, utilizado por diferentes exemplos do abnTeX2, ilustra o uso de
% comandos do abnTeX2 e de LaTeX.
% ---

\chapter{Desenvolvimento}\label{cap_exemplos}

\section{Objetivo}

O objetivo do presente relatório é apresentar uma solução simples que não demande um conhecimento técnico avançado, a partir dos dados disponíveis no SIM-DO será aplicado o processo de Extração, Transformação e Carregamento (ETL) da base de dados para a construção de um \textit{Dashboard} que possibilite o monitoramento das causas de morte, dado as limitações computacionais o escopo será limitado ao Estado de São Paulo na janela temporal entre os anos de 2010 e 2014.

\section{Metodologia}

A extração será feita por meio dos \textit{softwares} \textit{R} e \textit{RStudio}, onde foram utilizadas as funções da biblioteca desenvolvidas por \citeonline{saldanha2019microdatasus}, esta biblioteca foi criada para facilitar a extração dos dados disponibilizados pelo SUS, incluindo o sistema SIM, adicionalmente são disponíveis funções de pré processamento que tratam os valores presentes nas tabelas e, a partir dos dados extraídos, incluí informações de municipalidade em um formato utilizável pelas ferramentas de mapas presentes na maioria dos \textit{softwares} de visualização, toda a documentação e instrução de uso são disponíveis pela ferramenta de ajuda(documentação) da própria biblioteca.



