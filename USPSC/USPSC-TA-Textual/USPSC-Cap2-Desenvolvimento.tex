%% USPSC-Cap2-Desenvolvimento.tex 

% ---
% Este capítulo, utilizado por diferentes exemplos do abnTeX2, ilustra o uso de
% comandos do abnTeX2 e de LaTeX.
% ---

\chapter{Desenvolvimento}\label{cap_exemplos}

\section{Objetivo}

O objetivo do presente relatório é apresentar uma solução simples que não demande um conhecimento técnico avançado, a partir dos dados disponíveis no SIM-DO será aplicado o processo de Extração, Transformação e Carregamento (ETL) da base de dados para a construção de um \textit{Dashboard} que possibilite o monitoramento das causas de morte, dado as limitações computacionais o escopo será limitado ao Estado de São Paulo na janela temporal entre os anos de 2010 e 2022.

\section{Fundamentação Teórica}

O monitoramento de métricas para a saúde publica se mostra eficaz na produção de resultados, \citeonline{soares2019melhoria} realizou um estudo em 60 cidades, participantes do projeto Dados para a Saúde, que busca melhorar o diagnóstico de causas de morte por meio da cooperação de equipes de vigilância de óbitos. As cidades participantes do projeto Dados para a Saúde apresentaram melhores resultados após a investigação das Causas Externas Inespecíficas de Mortalidade (CEI), possibilitando uma análise detalhada da reclassificação para causas específicas, considerando sexo e faixas etárias.

Os resultados apresentados por \citeonline{soares2019melhoria} mostram a importância do monitoramento em conjunto, adicionalmente destaca a melhoria contínua nos sistemas relevantes, como o SIM-DO.

O monitoramento das causas de morte são importantes insumos para a construção de indicadores de efetividade dos serviços de saúde, \citeonline{malta2007causas} destaca a importância de uma lista de causas evitáveis para a construção de indicadores, porem a necessidade de levar em consideração fatores fora da cobertura dos sistemas de saúde, este problema necessita de uma abordagem mais complexa e fora do escopo deste trabalho.

A mortalidade por causas evitáveis também podem ser resultado de problemas sociais, que devem ser monitorados e enfrentados não só pelos sistemas de saúde, \citeonline{malta2021mortalidade} destaca que a mortalidade de adolescentes e adultos jovens apresentam tendência de alta por homicídios, e dados do  Fórum de Segurança Pública apontam uma sub-notificação, em especial entre jovens negros que residem em periferias urbanas, deste modo problemas de completude do SIM-DO são importantes para políticas públicas em um escopo geral, e não apenas na área da saúde.

Dado o exposto o presente trabalho se propõe a apresentar uma solução base, ao se utilizar dos dados presentes no SIM, e que possa incorporar molharias a medida em que os problemas apresentados sejam melhor compreendidos e a completude dos dados seja melhorada.



 

\section{Aplicação das disciplinas estudadas no Projeto Integrador}

Durante a execução do projeto, aplicamos conhecimentos e técnicas adquiridas em disciplinas como ETL, Arquitetura de Dados, \textit{Big Data Analytics}, Análise de Dados, Estatística I e II, Programação, \textit{BI} (\textit{Business Intelligence}) e \textit{Storytelling}. Adicionalmente o fluxo de trabalho foi organizado por meio da ferramenta \textit{Kanban}, a partir dos conceitos aprendidos em diversas disciplinas sobre metodologias de projeto.

Na fase de coleta e preparação de dados, utilizamos técnicas de limpeza e integração de dados aprendidas ao longo dessas disciplinas. Em seguida, na fase de análise, empregaremos técnicas estatísticas para extrair \textit{insights} significativos dos dados. Finalmente, na fase de visualização e interpretação dos resultados, utilizaremos habilidades de programação para desenvolver \textit{dashboards} interativos e relatórios.


\section{Metodologia}

A extração será feita por meio dos \textit{softwares} \textit{R} e \textit{RStudio}, onde foram utilizadas as funções da biblioteca desenvolvidas por \citeonline{saldanha2019microdatasus}, esta biblioteca foi criada para facilitar a extração dos dados disponibilizados pelo SUS, incluindo o sistema SIM, adicionalmente são disponíveis funções de pré processamento que tratam os valores presentes nas tabelas e, a partir dos dados extraídos, incluí informações de municipalidade, toda a documentação e instrução de uso são disponíveis pela ferramenta de ajuda da própria biblioteca.

Para a visualização dos resultados em um \textit{dashboard} os nomes dos estados e municípios foram retirados da tabela de Códigos dos municípios disponibilizada pelo Instituto Brasileiro de Geografia e Estatística (IBGE).

Adicionalmente foi elaborada uma tabela para mapear as causas de morte evitáveis com base na lista proposta por \citeonline{malta2007causas}, devido a limitação no escopo do trabalho as Causas mal-definidas e Acidentes de trânsito/transporte não foram incluídas.

De forma parcial a exploração dos dados foi feita por meio da linguagem \textit{Python} em um arquivo no formato \textit{Jupyter Notebook} de forma colaborativa por meio de versionamento pelo \textit{GitHub}, que possibilita a colaboração independente das ferramentas de preferência de cada colaborador. Após o carregamento dos dados na ferramenta Power BI um \textit{dashboard} foi criado para auxiliar a exploração dos dados e apresentação dos resultados.

