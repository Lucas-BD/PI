%% USPSC-Resumo.tex
\setlength{\absparsep}{18pt} % ajusta o espaçamento dos parágrafos do resumo		
\begin{resumo}
	\begin{flushleft} 
			\setlength{\absparsep}{0pt} % ajusta o espaçamento da referência	
			\SingleSpacing 
			\imprimirautorabr~~\textbf{\imprimirtituloresumo}. %\pageref{LastPage} p. 
			%Substitua p. por f. quando utilizar oneside em \documentclass
			\pageref{LastPage} f.
			\imprimirtipotrabalho. Big Data para Negócios~-~\textbf{\imprimirinstituicao}, \imprimirlocal, \imprimirdata. 
 	\end{flushleft}
\OnehalfSpacing 			
 O objetivo do presente trabalho é propor métodos simples de programação, \textit{Big Data} e visualização de dados para monitorar causas de morte a partir dos dados extraídos do Sistema de Informação sobre Mortalidade Declaração de Óbitos (SIM-DO) disponibilizado pelo Sistema Único de Saúde (SUS).
 Como trabalho parcial realizamos a extração e o tratamento dos dados por meio da linguagem de programação \textit{R}, utilizando o ambiente do \textit{RStudio} e bibliotecas específicas para o Sistema de Informações sobre Mortalidade (SIM).
 

 \textbf{Palavras-chave}: Relatório Técnico, Mortalidade, Causas de morte, Monitoramento, Visualização.
\end{resumo}