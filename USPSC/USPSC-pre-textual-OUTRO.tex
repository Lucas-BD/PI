%% USPSC-pre-textual-OUTRO.tex
%% Camandos para definição do tipo de documento (tese ou dissertação), área de concentração, opção, preâmbulo, titulação 
%% referentes ao Programa de Pós-Graduação o IQSC
\instituicao{Pastor En\'eas Tognini, Faculdade De Tecnologia}
\unidade{Pastor En\'eas Tognini}
\unidademin{Faculdade De Tecnologia}
\universidademin{Faculdade De Tecnologia}

%\notafolharosto{Vers\~ao original}
%Para versão original em inglês, comente o comando/declaração 
%     acima(inclua % antes do comando acima) e tire a % do 
%     comando/declaração abaixo no idioma do texto
%\notafolharosto{Original version} 
%Para versão corrigida, comente o comando/declaração da 
%     versão original acima (inclua % antes do comando acima) 
%     e tire a % do comando/declaração de um dos comandos 
%     abaixo em conformidade com o idioma do texto
%\notafolharosto{Vers\~ao corrigida \\(Vers\~ao original dispon\'ivel na Unidade que aloja o Programa)}
%\notafolharosto{Corrected version \\(Original version available on the Program Unit)}

% ---
% dados complementares para CAPA e FOLHA DE ROSTO
% ---
\universidade{Faculdade De Tecnologia}
\titulo{Monitoramento de Causas de Morte por meios interativos}
\titleabstract{Model for thesis and dissertations in \LaTeX\ using the USPSC Package}
\tituloresumo{Monitoramento de Causas de Morte por meios interativos}
\autor{
	Breno Juan da Fonseca Silva 
	\and \\
	Dayana Ingrid Carata Choque
	\and \\
	Gabriel Le\~ao Frigo
	\and \\
	Jackelyne Alicia Miranda Ramos
	\and \\
	Lucas Barbosa Defanti
	\and \\
	Lucas Matos Norbertino dos Santos
	\and \\
	Matheus Henrique Biano Neres
	\and \\
	Michelle Mieko Coelho Koga
	}
\autorficha{
	Silva, Breno Juan da Fonseca
	\and \\
	Choque, Dayana Ingrid Carata
	\and \\
	Frigo, 	Gabriel Le\~ao
	\and \\
	Ramos, Jackelyne Alicia Miranda
	\and \\
	Defanti, Lucas Barbosa
	\and \\
	Santos, Lucas Matos Norbertino dos
	\and \\
	Neres, Matheus Henrique Biano
	\and \\
	Koga, Michelle Mieko Coelho
	}
\autorabr{
	CHOQUE, D.I. Catarata \textit{et al.}
}

\cutter{S856m}
% Para gerar a ficha catalográfica sem o Código Cutter, basta 
% incluir uma % na linha acima e tirar a % da linha abaixo
%\cutter{ }

\local{S\~ao Paulo - SP}
\data{2024}
% Quando for Orientador, basta incluir uma % antes do comando abaixo
\renewcommand{\orientadorname}{Orientador:}
% Quando for Coorientadora, basta tirar a % utilizar o comando abaixo
%\newcommand{\coorientadorname}{Coorientador:}
\orientador{Prof. Bruno Monserrat Perillo.}
\orientadorcorpoficha{orientador Bruno Monserrat Perillo.}
\orientadorficha{Perillo, Bruno Monserrat, orient}
%Se houver co-orientador, inclua % antes das duas linhas (antes dos comandos \orientadorcorpoficha e \orientadorficha) 
%          e tire a % antes dos 3 comandos abaixo
%\coorientador{Prof. Dr. Jo\~ao Alves Serqueira}
%\orientadorcorpoficha{orientadora Elisa Gon\c{c}alves Rodrigues ;  co-orientador Jo\~ao Alves Serqueira}
%\orientadorficha{Rodrigues, Elisa Gon\c{c}alves, orient. II. Serqueira, Jo\~ao Alves, co-orient}

\notaautorizacao{AUTORIZO A REPRODU\c{C}\~AO E DIVULGA\c{C}\~AO TOTAL OU PARCIAL DESTE TRABALHO, POR QUALQUER MEIO CONVENCIONAL OU ELETR\^ONICO PARA FINS DE ESTUDO E PESQUISA, DESDE QUE CITADA A FONTE.}
\notabib{Ficha catalogr\'afica elaborada pela Biblioteca da Unidade USP, com os dados fornecidos pelo(a) autor(a)}

\newcommand{\programa}[1]{

% TCCOUTRO ==========================================================================
\ifthenelse{\equal{#1}{DOUTRO}}{
    \area{Nome da \'Area}
	\tipotrabalho{Relat\'orio T\'ecnico-Cient\'ifico}
	\tipotrabalhoabs{Technical Report}
	%\opcao{Nome da Opção}
    % O preambulo deve conter o tipo do trabalho, o objetivo, 
	% o nome da instituição e a área de concentração 
	\preambulo{Relatório Técnico-Científico apresentado na disciplina de Projeto Integrador para o curso de Big Data para Negócios da Faculdade de Tecnologia Ipiranga “Pastor Enéas Tognini” (FATEC).}
	\notaficha{Relatório Técnico-Científico apresentado na disciplina de Projeto Integrador para o curso de Big Data para Negócios da Faculdade de Tecnologia Ipiranga “Pastor Enéas Tognini” (FATEC).}
    }{
% MOUTRO ===========================================================================
\ifthenelse{\equal{#1}{MOUTRO}}{
    \area{Nome da \'Area}
	\tipotrabalho{Disserta\c{c}\~ao (Mestrado)}
	\tipotrabalhoabs{Dissertation (Master)}
	%\opcao{Nome da Opção}
    % O preambulo deve conter o tipo do trabalho, o objetivo, 
	% o nome da instituição e a área de concentração 
	\preambulo{Relatório Técnico-Científico apresentado na disciplina de Projeto Integrador para o curso de Big Data para Negócios da Faculdade de Tecnologia Ipiranga “Pastor Enéas Tognini” (FATEC).}
	\notaficha{Relatório Técnico-Científico apresentado na disciplina de Projeto Integrador para o curso de Big Data para Negócios da Faculdade de Tecnologia Ipiranga “Pastor Enéas Tognini” (FATEC).}
    }{
% Outros
    \tipotrabalho{Relat\'orio T\'ecnico-Cient\'ifico}
	\tipotrabalhoabs{Technical Report}
	%\area{Nome da \'Area}
	%\opcao{Nome da Op\c{c}\~ao}
	% O preambulo deve conter o tipo do trabalho, o objetivo, 
	% o nome da instituição e a área de concentração 
	\preambulo{Relatório Técnico-Científico apresentado na disciplina de Projeto Integrador para o curso de Big Data para Negócios da Faculdade de Tecnologia Ipiranga “Pastor Enéas Tognini” (FATEC).}
	\notaficha{Relatório Técnico-Científico apresentado na disciplina de Projeto Integrador para o curso de Big Data para Negócios da Faculdade de Tecnologia Ipiranga “Pastor Enéas Tognini” (FATEC).}
    }}}
				