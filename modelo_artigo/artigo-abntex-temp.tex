%% Este pequeno template te ajuda a compilar um artigo simples.
%% É uma versão mínima do modelo canônico de artigo científico do AbnTeX.
%% Mais informações veja o site: https://www.abntex.net.br/

%% Testamos em Linux, Mac OSX e Windows em Julho de 2019.

\documentclass[
	article,
	12pt,
	oneside,
	a4paper,
	english,
	brazil
	]{abntex2}

\usepackage[brazil]{babel}
\usepackage{lmodern}
\usepackage[T1]{fontenc}
\usepackage[utf8]{inputenc}
\usepackage{indentfirst}
\usepackage{color}
\usepackage{graphicx}
\usepackage[brazilian,hyperpageref]{backref}
\usepackage[alf]{abntex2cite}

\hypersetup{
  		colorlinks=true,
    	citecolor=black       % cor da citação.
}

\titulo{Seu título}
\tituloestrangeiro{}       % NÃO REMOVA. Usar abntex2 como artigo requer que o título
                             % estrangeiro seja utilizado. Mantenha vazio ou adicione o título em inglês.
\autor{Seu Nome}



\begin{document}
\maketitle



\begin{resumoumacoluna}
 Comece seu resumo... \\
 
 \textbf{Palavras-chave}: latex. \abnTeX. editoração de texto.
\end{resumoumacoluna}

\textual
\section{Introdução}

A maior vantagem do \abnTeX é as citações no formato da abnt. Por exemplo, a literatura entende que isso salvou a vida dos estudantes universitários \cite{Abreu1999}. Vou vender minha arte na praia.

\section{Revisão de Literatura}
Segundo \citeonline{Abreu1999} isso vai te fazer publicar Qualis A. Não use citação direta, mas

\begin{citacao}
A maior vantagem do \abnTeX é as citações no formato da abnt. Por exemplo, a literatura entende que isso salvou a vida dos estudantes universitários \cite{Abreu1999}
\end{citacao}


\section{Resultados}

\bibliography{referencias}

\end{document}
